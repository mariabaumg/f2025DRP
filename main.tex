%Using pdfLaTeX

\documentclass[10pt,compress]{beamer}

\usetheme{Madrid}
% \definecolor{rutgersScarlet}{RGB}{204,0,51}
\definecolor{rutgersScarlet}{RGB}{115,79,150}
\definecolor{grey}{RGB}{150,150,150}
\usecolortheme[named=rutgersScarlet]{structure}
\setbeamertemplate{navigation symbols}{}
\setbeamertemplate{title page}[default][colsep=-0bp,rounded=true]
\setbeamerfont{structure}{family=\rmfamily,series=\bfseries}
\usefonttheme[stillsansseriftext]{serif}
\setbeamertemplate{blocks}[default]
\setbeamercolor{block body}{bg=white!30,fg=black}
\usepackage{amsthm} 
% \newtheorem{note}{Note}[section]


\setbeamertemplate{itemize items}[default]
\setbeamertemplate{enumerate items}[default]
\setbeamersize
{
    text margin left=0.7cm,
    text margin right=1cm
}

\usepackage{amsfonts}
\usepackage{amsmath}
\usepackage{amssymb}
\usepackage{amsthm}
\usepackage{multicol}
\usepackage{physics}
\usepackage{caption}
\usepackage{graphicx,comment}
\usepackage[english]{babel}
\usepackage{graphicx}
\usepackage{rotating}
\usepackage{hyperref}
\usepackage{enumerate}
\usepackage{tikz}
\usepackage{pgfplots}
\usepackage{bm}
\usepackage{faktor}
\usepackage{mathtools}
\usepackage{multicol}


\usepackage{xparse}

\NewDocumentEnvironment{proof*}{o}
  {%
    \IfNoValueTF{#1}
      {\begin{proof}}
      {\begin{proof}[#1]}%
  }
  {%
    \renewcommand{\qedsymbol}{}%
    \end{proof}%
  }

\graphicspath{ {./images/} }

\title[]{Analytic Number Theory: Modular Forms}

\author[]{Maria Baumgartner}

\date[]{December 11, 2025}

\institute[]{Rutgers University}

\begin{document}

%Title Slide
\begin{frame}
\titlepage
\end{frame}

\begin{frame}{Motivation}
    \begin{itemize}
        \item Prove the functional equation for L-functions associated to Hecke eigenforms
    \end{itemize}
\end{frame}



\begin{frame}{Outline}
    \tableofcontents
\end{frame}

\section{Classical Analytic Number Theory}
\begin{frame}{Transforms}
    \begin{definition}[Fourier Transform]
        For integrable function $f:\mathbb{R}\to\mathbb{C}$, the Fourier Transform is defined by
        \[
            \tilde{f}(s) = \int_\mathbb{R} f(t)e^{-2\pi ist}\ dt.
        \]
    \end{definition}
    % \pause
    \begin{definition}[Mellin Transform]
        For integrable function $f:\mathbb{R}\to\mathbb{C}$, the Mellin Transform is defined by
        \[
            \left(\mathcal{M}f\right)(s) = \int_0^\infty t^{s-1}f(t)\ dt.
        \]
    \end{definition}
\end{frame}


\begin{frame}{Riemann Zeta Function}
\begin{definition}%[Riemann Zeta Function]
For $\Re(s)>1$,
\begin{align*}
\zeta(s) &= \sum_{n=1}^\infty \frac{1}{n^s}\\
  &= \prod_{p} \frac{1}{1-1/p^s}, \text{ by Lemma (Euler Product).}
\end{align*}
\end{definition}
\pause
\begin{itemize}
    \item May be used to show infinitude of primes.
\end{itemize}
\end{frame}

%-------------------------------------------------------------
\begin{frame}{Dirichlet Characters}
    \begin{definition}[Dirichlet Characters]
        The function $\chi:\mathbb{Z}\to\mathbb{C}$ with associated modulus $q$ satisfying
        \begin{itemize}
            \item $\chi(ab)=\chi(a)\chi(b)$ (multiplicativity)
            \pause
            \item $\chi(a)=0$ iff $(a,q)\neq 1$
            \pause
            \item $\chi(a+q)=\chi(a)$ (i.e. periodic mod $q$)
        \end{itemize}
    \end{definition}
    \pause
    \begin{definition}[Primitive Dirichlet Characters]
        Note that $\chi$ has \textit{quasiperiod} $d$ if $\chi(m)=\chi(n)$ for all $m$ and $n$ coprime to $q$ such that $m\equiv n \mod d$.
        \pause
        
        \medskip

        The smallest such $d$ is the \textit{conductor}.
        \pause
        
        \medskip
        
        If the conductor equals $q$, then $\chi$ is primitive.
    \end{definition}
\end{frame}

\begin{frame}{Orthogonality of Dirichlet Characters}
    \begin{definition}[$\chi_0$]
    \begin{align*}
    \chi_0(n) &= \begin{cases}
    1 & (n,q)=1,\\
    0 & \text{otherwise.}
    \end{cases}
    \end{align*}
    \pause
    \end{definition}
        \begin{lemma}[Orthogonality of Dirichlet Characters]
        For a character $\chi$ with modulo $q$,
        \[
            \sum_{m=0}^{q-1}\chi(m) = \begin{cases}
            \phi(q) & \chi=\chi_0,\\
            0 & \chi\neq \chi_0.
            \end{cases}
        \]
        \end{lemma}
\end{frame}

\begin{frame}{Dirichlet L-function}
        \begin{definition}[Dirichlet L-function]
        For Dirichlet Character $\chi$, we define $L(\chi, s)$ for $\Re(s)>1$ as
        \begin{align*}
            L\left(\chi,s\right) &= \sum_{n=1}^\infty \frac{\chi(n)}{n^s}\\
            &= \prod_{p} \frac{1}{1-\chi(p)/p^s}, \text{ by Lemma (Euler product).}
        \end{align*}
    \end{definition}
\end{frame}

%-------------------------------------------------------------
\begin{frame}{Nonvanishing of $L(\chi,1)$}
    \begin{theorem}%[Nonvanishing of $L(\chi,1)$ implies Dirichlet's Theorem]
        If $L(\chi,1)\neq 0$ for all $\chi\neq\chi_0$, then there exist infinitely many primes.
        
    \end{theorem}
    \pause
        \begin{proof*}{Proof Sketch.}
        \begin{itemize}
        \item We start with
        \[
            L\left(\chi,s\right) = \prod_{p} \frac{1}{1-\chi(p)/p^s}
        \]
        \pause
        \item Then take the $\log$ of both sides,
        \[
            \log L(\chi,s) = \sum_p\sum_{1\leq n}\frac{\chi(p^n)}{np^{ns}}.
        \]
        \end{itemize}
    \end{proof*}
\end{frame}

\begin{frame}{Proof ct.}
\begin{proof*}{Proof Sketch ct.}
\begin{itemize}
    \item Considering
        \[
        \frac{1}{\phi(q)}\sum_{\chi\in \chi_q}\bar{\chi}(a)\log L(\chi,s)
        \]
        \pause
        we plug in what we got for $\log L(\chi,s)$, getting
        \[
        \sum_p\sum_{1\leq n}\frac{1}{np^{ns}}\left(\frac{1}{\phi(q)}\sum_{\chi\in\chi_q} \bar{\chi}(a)\chi(p^n)\right).
        \]
        \end{itemize}
\end{proof*}
\end{frame}


\begin{frame}{Proof ct.}
\begin{proof*}{Proof Sketch ct.}
\begin{itemize}
\item We then break up the summation
\[
\underbrace{\sum_{\substack{p \\ p \equiv a \pmod q}} \frac{1}{p^s}}_{(1)}
\;+\;
\underbrace{\sum_{2\le n}\left(\sum_{\substack{p \\ p^n\equiv a \pmod q}} \frac{1}{n p^{ns}}\right)}_{(2)}.
\]
\pause
\item We are able to bound (2). Thus, we must show our original expression is unbounded to prove (1) is unbounded.
\end{itemize}
\end{proof*}
\end{frame}

\begin{frame}{Proof ct.}
\begin{proof}{Proof Sketch ct.}
\begin{itemize}
\item In particular, we show
\[
        \lim_{s\to 1^+}\frac{1}{\phi(q)}\sum_{\chi\in \chi_q}\bar{\chi}(a)\log L(\chi,s) = \infty
        \]
since this would imply (1) diverges to $\infty$.
\pause
\item Breaking up the sum, we observe $\chi_0$ produces a $\log(\zeta(s))$ which goes to $\infty$ as $s\to 1^+$. All other $\chi\neq\chi_0$ contribute only bounded terms if $L(\chi,1)\neq 0$.
\pause
\item Using orthogonality and Euler products, we can conclude with the hypothesis that there are infinitely many primes.
\end{itemize}
\end{proof}
\end{frame}



\begin{frame}{Further Results}
\begin{itemize}
\item For Complex $\chi$, $L(\chi,1)\neq 0$ (Lemma).

\medskip

\item For Real $\chi$, $L(\chi,1)\neq 0$ (Lemma).
\end{itemize}
\end{frame}



\section{Modular and Cusp Forms}

\begin{frame}{Modular Forms/Motivation}
We've built up some analytic tools.

\bigskip
Why analyze Modular Forms?
\begin{itemize}
    \item We can choose matrices in $SL_2(\mathbb{Z})$ to yield interesting results,
    \item Possess Fourier expansions $f(z)=\sum a_n e^{2\pi i n z}$.
\end{itemize}
\end{frame}


\begin{frame}{Modular Forms}
\begin{definition}[Slash Operator]
For $\gamma\in \text{GL}_2(\mathbb{R})$,
\begin{align*}
f\vert_k \gamma(z) := \det \gamma^{k/2} j(\gamma,z)^{-k}f(\gamma z)
\end{align*}
\textbf{Note:} $\text{GL}_2(\mathbb{R}) = \{A\in M_{2\times 2}(\mathbb{R}) \mid \det(A)\neq 0\}.$
\end{definition}

\pause
\begin{definition}[Modular Form]
Note that $\mathcal{H}=\{z\in\mathbb{C}\mid z\text{ has positive imaginary part}\}$. A Modular Form is a function $f:\mathcal{H}\to \mathbb{C}$ associated to a weight $k$ such that
\begin{align*}
f\vert_k \gamma(z) = f(z)
\end{align*}
where $\gamma\in\Gamma$ and $\Gamma$ is a subgroup of $\text{SL}_2$.
\pause
\textbf{Note:}
\begin{itemize}
\item $\text{SL}_2(\mathbb{Z}) = \{A\in M_{2\times 2}(\mathbb{Z}) \mid \det(A)=1\}$.
\item $\gamma z = (az+b)/(cz+d)$ and $j(\gamma,z)=cz+d$.
\end{itemize}
\end{definition}
\end{frame}




\begin{frame}{Cusp Forms}
\begin{definition}[Cusp Form]
Function $f:\mathcal{H}\to \mathbb{C}$ in modular form and vanishes at $i\infty$.
\begin{itemize}
\item Vanish at every cusp
\end{itemize}
\end{definition}

\pause
Further:
\begin{itemize}
    \item If $f$ has weight $k$, then $f$ is a weight $k$ cusp form iff $f(\infty)=0$ (Lemma).
    \pause
    \item If $f$ has fourier expansion:
          \[
          f(z) = \sum_{0\leq m} a_m e^{2\pi i m z}.
          \]
          then $a_0=0$.
\end{itemize}
\end{frame}


\begin{frame}{Eisenstein Series and Delta Function}
    \begin{definition}[Eisenstein series]
\begin{align*}
E_k(z) &= \sum\sum_{(m,n)\neq(0,0)} \frac{1}{(mz+n)^k}\\
&= 1 - \frac{2k}{B_k}\sum_{1\leq m} \sigma_{k-1}(m) e^{2\pi i mz}, \text{ by Lemma (Fourier Expansion)}.
\end{align*}
\end{definition}

\end{frame}

\begin{frame}{Eisenstein Series and Delta Function}
\begin{definition}[$\Delta$-Function]
\[
\Delta(z)=\frac{E_4(z)^3 - E_6(z)^2}{1728}.
\]
\end{definition}

\pause
Note:
\begin{itemize}
    \item $\Delta$ is a weight 12 cusp form.
    \pause
    \item Non-vanishing of $\Delta$ on $\mathcal{H}$ except for a simple zero at $i\infty$ can be shown via the Valence Formula (Lemma).
\end{itemize}
\end{frame}


\begin{frame}{Structure of $M_k$}
In the space $M_k$ of modular forms of weight $k$, the following properties hold:
\begin{itemize}
\item If $k<0$ or odd, $M_k=\emptyset$.
\item If $k=0$, $M_k=\mathbb{C}$.
\item If $k=2$, then $M_k=\emptyset$.
\item If $4\leq k\leq 10$ is even, $M_k=\mathbb{C}\cdot E_k$.
\item If $k\geq 12$, then $M_k=\Delta M_{k-12}\oplus \mathbb{C}E_{k}$
\end{itemize}

\medskip
\pause
Further,
\[
M_k = \Delta M_{k-12} \;\oplus\; \mathbb{C} E_k.
\]

\pause
\medskip

Consequences:
\begin{itemize}
    \item All modular forms are built from $E_4$, $E_6$, and $\Delta$.
    \item $M_{k+12} \in \mathbb{C}[E_4,E_6]$ (Proven via induction.)
\end{itemize}

\end{frame}




\section{Hecke Operators}
\begin{frame}{Hecke Operators}
\begin{definition}[Hecke Operator]
Acts on function $f\in M_k(\Gamma_0(N))$ as
\begin{align*}
(T_n f)(z) = n^{k/2-1} \sum_{\delta\in \Delta_n^N} f\vert_k \delta(z).
\end{align*}
\textbf{Remark:} can think of Hecke operators as an averaging operator.
\end{definition}
\pause
Hecke operators have the property:
\[
T_m T_n = T_{mn} \quad \text{if } (m,n)=1.
\]

\pause
If $f$ has fourier expansion
\[
f(z) = \sum_{1\leq m}c_m e^{2\pi imz}
\]
with $c_1=1$ and there exists $\lambda_n$ s.t. $T_n f =\lambda_n f$ for all $n$, then
\[
c_{mn}=c_mc_n \quad \forall (m,n)=1.
\]
\end{frame}


\begin{frame}{How Hecke Operators Act}
\begin{lemma}
    For $f$ with Fourier expansion $f(z)=\sum_{0\leq m} c_m e^{2\pi imz}$ then,
\[
T_n f(z)
= \sum_{0\leq m}
    \left( \sum_{d\mid (m,n)} \chi_0^N(d) d^{k-1} c_{mn/d^2} \right)
      e^{2\pi i mz}.
\]
\end{lemma}

\pause
Key consequences:
\begin{itemize}
    \item $T_n$ preserves cusp forms.
    \item Hecke operators commute and are self-adjoint under the Petersson inner product.
\end{itemize}

Thus $S_k(\Gamma_0(N))$ has a basis of simultaneous eigenforms.
\end{frame}

\section{Hecke Bound}
\begin{frame}{Hecke Bound (Used in L-function Analysis)}
Note that, for a cusp form of weight $k$:
\[
|a_n| \ll n^{k/2}.
\]

\pause
Why this matters:
\begin{itemize}
    \item Ensures L-function for cusp forms, $L(f,s)=\sum a_n n^{-s}$, converges absolutely for $\Re(s)>1+\frac{k}{2}$.
    \item Key part of deriving the functional equation for the L-functions associated with cusp forms.
\end{itemize}

\end{frame}


\section{Newforms}

\begin{frame}{Newforms}
\begin{itemize}
    \item For level $N$, cusp forms decompose into $S_k^{\mathrm{old}}$ and $S_k^{\mathrm{new}}$
    
    \medskip

    \item A level $N$ \textbf{newform} is an $f\in S_k^{\mathrm{new}}$ such that $f$ is a normalized simultaneous eigenform for all $T_n$ with $(n,N)=1$.
    
    \medskip


    \pause
    \item Key property: \textbf{multiplicity one}.
          \[
          f,g \in S_k^{\mathrm{new}}(\Gamma_0(N)),\ 
          \lambda_n(f)=\lambda_n(g)\quad \forall (n,N)=1 \quad 
          \Rightarrow f=g.
          \]
\end{itemize}
\end{frame}

%-------------------------------------------------------------
\begin{frame}{Newforms}
\begin{theorem}
A level $N$ newform $f$ is a normalized simultaneous eigenform of all Hecke operators $T_n$.
\end{theorem}
\begin{itemize}
\item If $f$ is a newform and $T_n f = \lambda_n f$ for $(n,N)=1$, then $T_m f = \lambda_m f \quad \text{for all } m.$
\item Note that proof makes use of multiplicity one principle.
\end{itemize}
\end{frame}

%-------------------------------------------------------------
\section{Fricke Involution}
\begin{frame}{The Fricke Involution}
Define by letting $Wf = f\vert_k \omega$ where
\[
\omega = 
\begin{pmatrix}
0 & -1/\sqrt{N}\\
\sqrt{N} & 0
\end{pmatrix}.
\]

\pause
\textbf{Key facts:}
\begin{itemize}
    \item For $\gamma = \begin{pmatrix} a & b \\ cN & d \end{pmatrix} \in \Gamma_0(N)$,
    \[
    \omega \gamma = \gamma' \omega
        \]
    for some $\gamma' \in \Gamma_0(N)$.
    \pause
    \item Fricke Involution preserves modular and cusp forms, i.e. for $f \in M_k(\Gamma_0(N))$ or $\in S_k(\Gamma_0(N))$,
    \[
    (Wf)\vert_k \gamma = Wf.
    \]
    \pause
    \item Fricke involution commutes with Hecke operators
    \[
    WT_n = T_nW
    \]
\end{itemize}
\end{frame}

%-------------------------------------------------------------
\begin{frame}{Fricke and Hecke Commute}
\textbf{The operators commute:}
\[
T_n W_N = W_N T_n.
\]

Consequences:
\begin{itemize}
    \item If $f\in S_k^{\text{new}}(\gamma_0(N))$ is a Hecke eigenform, then $f=cg$ where $g$ is a newform.
    \pause
    \item We can evaluate,
    \[
    T_n Wg = \lambda_n Wg.
    \]
    So $Wg$ is a Hecke eigenform.
    \pause
    \item Since $Wg=wh$ where $h$ is a newform, we get
    \[
    T_n h = \lambda_n h,
    \]
    \pause
    \item So $g,h$ have same Hecke eigenvalues so $g=h$. Thus, $Wf=w f$ for some $w\in \mathbb{C}$.
\end{itemize}
\end{frame}

\begin{frame}{Fricke and Hecke Commute}
\begin{lemma}
If $f$ is a Hecke eigenform then $f$ is an eigenform of the Fricke involution. Specifically $Wf=\pm f$.
\end{lemma}

\end{frame}

%-------------------------------------------------------------
\section{The Functional Equation}
\begin{frame}{Setting Up the Functional Equation}
For a cusp form $f(z) = \sum a_n e^{2\pi i n z}$, define its $L$-function:
\[
L(f,s) = \sum_{n\ge1} a_n n^{-s}.
\]

Completed $L$-function (scaled by $N^{s/2}$):
\[
\Lambda(f,s) = N^{s/2} (2\pi)^{-s}\Gamma(s)L(f,s).
\]

\pause
Use Mellin transform:
\[
\Lambda(f,s) = N^{s/2} \int_0^\infty f(it)\, t^{s} \frac{dt}{t}.
\]
\pause
Split integral at $t = 1/\sqrt{N}$ (allowed by Fricke),
\[
N^{s/2}\left(\int_0^{1/\sqrt{N}} f(it)\, t^{s} \frac{dt}{t} + \int_{1/\sqrt{N}}^\infty f(it)\, t^{s} \frac{dt}{t} \right).
\]
\end{frame}

%-------------------------------------------------------------
\begin{frame}{Using Fricke and Final Equation}
Compute:
\[
\int_0^{1/\sqrt{N}} f(it)\,t^{s} \frac{dt}{t}
= \pm i^{\,k} N^{\frac{k}{2}-s}
   \int_{1/\sqrt{N}}^\infty f(it)\,t^{k-s} \frac{dt}{t}.
\]

\pause
\medskip

Combining with the second piece yields:
\[
\boxed{\Lambda(f,s)
= \pm\, i^{k}\, \Lambda(f, k-s).}
\]

\textbf{This is the functional equation for L-functions associated with Hecke eigenforms.}
\end{frame}


\begin{frame}{Acknowledgements}
Much thanks to the DRP team and to my wonderful mentor Preston Tranbarger!

\medskip

\href{https://prestontranbarger.github.io/f/qualExams/OralQualNotes.pdf}{Click here for the notes used!}

\medskip

\centering
\textbf{Thank you!}
\end{frame}


\end{document}
